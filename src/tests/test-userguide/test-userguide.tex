% arara: lualatex
% arara: bib2gls: { group: on }
% arara: lualatex: { shell: on }
% arara: bib2gls: { group: on } if found ("log", "Glossary entry `sym.")
% arara: lualatex if found ("log", "Rerun to")

\documentclass[titlepage=false,oneside,
 fontsize=12pt,captions=tableheading]{scrbook}

%\usepackage[autooneside=false]{scrlayer-scrpage}

\usepackage
 [debug=showwrgloss]
{nlctuserguide}

\renewcommand*{\thispackagename}{foobarsty}

\nlctuserguidegls
{
% commands
  \gcmd{foo\-bar\-cs}
  {%
    \syntax{\oargm{options}\margm{text}}%
    \desc{a sample command}
    \providedby{\sty{foobarsty} v1.0+}
    \note{a sample note}
  }
  \gcmd{cs\-with\-starred\-version}
  {%
    \syntax{\oargm{options}\margm{text}}%
    \desc{another sample command}
    \field{modifiers}{*}
  }
  \gcmd{cs\-with\-star\-and\-plus\-version}
  {%
    \syntax{\oargm{options}\margm{text}}%
    \desc{another sample command with star or plus versions}
    \field{modifiers}{*,+}
  }
  \gcmd{cs\-with\-starred\-other\-syntax}
  {%
    \syntax{\oargm{options}\margm{text}}%
    \desc{another sample command}
    \field{modifiers}{*}
  }
  \gcmd{cs\-with\-starred\-other\-syntax*}
  {%
    \syntax{\oargm{options}\margm{text}\margm{another arg}}%
    \desc{starred command}
  }
  \gcmd{deprecated\-cmd}
  {\deprecated
    \syntax{\margm{text}}
    \desc{deprecated command}
  }
  \gcmd{incompatible\-cmd}
  {\banned
    \syntax{\margm{text}}
    \desc{this command is incompatible with \sty{foobarsty}}
  }
  \gcmd{ab}
  {%
    \desc{minor command}
  }
  \gcmd{yz}
  {%
    \desc{another minor command}
  }
% package
  \gpkg{foobarsty}
  {%
    \note{or \code{\csfmt{usepackage}[foobar]\marg{baz}}}%
    \syntax{\meta{options}}
  }
% package options
  \gstyopt{styopt}
  {
    \inpackage{foobarsty}
    \syntax{\meta{value}}
    \defval{default}
    \desc{sample package option}
  }
% package option value
  \goptval{styopt}{value1}
  {
    \desc{some value}
  }
  \goptval{styopt}{value2}
  {
    \desc{another value}
  }
  \goptval{styopt}{default}
  {
    \desc{default value}
  }
% command options
  \gcsopt{foo\dhyphen bar}%
  {%
    \parent{foobarcs}%
    \desc{a sample valueless option}
  }%
  \gcsboolopt{foo\dhyphen bar\dhyphen false}
  {%
    \parent{foobarcs}%
    \initval{false}
    \desc{a sample boolean option that's initially false}
  }
  \gcsboolopt{foo\dhyphen bar\dhyphen true}
  {%
    \parent{foobarcs}%
    \initval{true}
    \desc{a sample boolean option that's initially true}
  }
  \gcsopt{foo\dhyphen bar\dhyphen val}
  {%
    \parent{foobarcs}
    \syntax{\meta{value}}
    \desc{a sample key=value option}
  }
  % option values
  \goptval{foo-bar-val}{optionvalue}
  {
    \desc{some allowed value}
  }
  \goptval{foo-bar-val}{deprecatedvalue}
  {
    \deprecated
    \field{alias}{optval.foo-bar-val.optionvalue}
    \desc{some deprecated value}
  }
  \gidx{commonoption}{\name{common options}\desc{some common options}}
  \gopt{some-option}{\parent{idx.commonoption}\desc{an option common to several commands}}
  \gopt{some-option2}{\parent{idx.commonoption}
    \syntax{\meta{boolean}}\defval{true}\initval{false}
    \desc{another option common to several commands}}
  % environments
  \genv{foobarenv}{\syntax{\margm{text}}\desc{a sample environment}}
  % counters
  \gctr{foobarctr}{\desc{a sample counter}}
  % applications
  \gapp{foobar\dhyphen cli}%
  {%
    \syntax{\meta{options} \meta{aux-file}}%
    \desc{a sample \idx{cli}}
  }
  \gapp{foobar\dhyphen nodesc}{}%
  % CLI switches
  \glongswitch{long\dhyphen switch}{\inapp{foobar-cli}}
  \gshortswitch{s}{\inapp{foobar-cli}\field{alias}{switch.long-switch}}
  % abbreviations
  \gabbr{html}{HTML}{hypertext markup language}{}
  \gacr{xml}{XML}{extensible markup language}{}
  \gtermacr{ascii}{ASCII}{American Standard Code for Information Interchange}
  {\field{description}{a single-byte character encoding}}
  \gtermabbr{cli}{CLI}{command-line interface}%
  {%
    \desc{an application that doesn't have a graphical user
     interface. That is, an application that doesn't have any windows,
     buttons or menus and can be run in
     \dickimawhref{latex/novices/html/terminal.html}{a command
     prompt or terminal}}%
  }%
  % terms
  \gterm{someterm}{\name{some term}\desc{an example term, see
   \sectionref{ch:sample}}}
  % indexed symbols
  \gpunc{sym.hash}{\name{\code{\#}}}
  % nbsp
  \gidx{nbsp}{\name{non-breaking space (\code{\textasciitilde})}
    \field{symbol}{\code{\textasciitilde}}
  }
  \gpunc{sym.nbsp}{\name{\code{\textasciitilde} (non-breaking space)}
    \field{see}{idx.nbsp}
  }
  \gpunccmd{cs.bksl}{\glsbackslash}{}% \\
  \gpunc{sym.bksl}{\name{\code{\glsbackslash} (literal backslash)}
    \field{symbol}{\code{\glsbackslash}}
  }
}

\title{Sample Document}
\author{Nicola Talbot}

\begin{document}
\maketitle

\begin{important}
Something important.
\end{important}

\begin{information}
Some information.
\end{information}

\begin{abstract}
An abstract.
\end{abstract}

\frontmatter
\tableofcontents
\listofexamples

\mainmatter
\chapter{Sample}\label{ch:sample}

Cross-references: \sectionref{sec:defs}, \tableref{tab:sample}.

\begin{table}[htbp]
\caption{Sample Table}
\label{tab:sample}
\centering
\begin{tabular}{llp{0.5\textwidth}}
\bfseries Column 1 &
\bfseries Column 2 &
\bfseries Column 3\\
\inlineglsdef{ab} &
\inlineglsdef{yz} &
Some text that requires a wider column and possibly multiple lines
\end{tabular}
\end{table}

First use: \idx{html}, next use: \idx{html}.
First use: \idx{xml}, next use: \idx{xml}.
\Idx{someterm}. Something about a \idx{nbsp} and \idx{sym.hash}.
A footnote\footnote{Footnote text}. A double backslash \gls{cs.bksl}
(control symbol) verses a literal backslash \sym+{bksl}.

\section{Semantic Markup}\label{sec:semantic}

Meta: \meta{text} and \qt{quoted} and \qtt{quoted-tt}.
File: \metafilefmt{filename}{tag}{.txt}, 
\metametafilefmt{file}{tag1}{name}{tag2}{.txt}. 

\section{Definitions}\label{sec:defs}

\pkgdef{foobarsty}

Package options.

\optiondef{styopt}
This option may take the following values.

\optionvaldef{styopt}{value1}
Does something.

\optionvaldef{styopt}{value2}
Does something else.

\optionvaldef{styopt}{default}
Some default setting.

Command: \gls{foobarcs} defined in \sty{foobarsty}.

\cmddef{foobarcs}

This command has the following options:
\optiondef{foo-bar}
This doesn't take a value.

\optiondef{foo-bar-false}
This takes a boolean value that is initially false.

\optiondef{foo-bar-true}
This takes a boolean value that is initially true.

\optiondef{foo-bar-val}
This takes a value, which may be one of the following:

\optionvaldef{foo-bar-val}{optionvalue}

\optionvaldef{foo-bar-val}{deprecatedvalue}
This is a deprecated synonym of
\opteqvalref{foo-bar-val}{optionvalue} so don't use it in new
documents.

Command with starred version:

\cmddef{cswithstarredversion}

Command with starred and plus version:

\cmddef{cswithstarandplusversion}

Command with starred version that has different syntax:

\cmddef{cswithstarredothersyntax}

Something about \gls{cswithstarredothersyntax*}.

A deprecated command:

\cmddef{deprecatedcmd}

Don't use \gls{incompatiblecmd}, which is incompatible with
\sty{foobarsty}.
\begin{badcodebox}
\gls{incompatiblecmd}
\end{badcodebox}

A \idx{cli} application \app{foobar-cli} is invoked with:
\appdef{foobar-cli}
Where \meta{options} may be:
\switchdef{long-switch}
\switch{s} is a short synonym.

Some common options.
\optiondef{some-option}

\optiondef{some-option2}

An environment:
\envdef{foobarenv}

A counter:
\ctrdef{foobarctr}

\section{Section About \stytext{foobarsty}}

\gls{foobarcs} is defined as:
\begin{compactcodebox}
\comment{a comment}
\cmd{newcommand}\marg{\gls{foobarcs}}\oarg{1}\marg{\cmd{emph}\marg{\#1}}
\end{compactcodebox}
Plus modifier: \sty+{foobarsty}.
Star modifier: \sty*{foobarsty}.

Plus modifier: \gls+{pkg.foobarsty}.
Star modifier: \gls*{pkg.foobarsty}.

Some code:
\begin{codebox}
\gls{foobarcs}\marg{text}
\end{codebox}
This emphasizes \code{text}:
\begin{resultbox}
\emph{text}
\end{resultbox}

Side-by-side:
\begin{coderesult}
\comment{another comment}
\gls{foobarcs}\marg{text}
\tcblower
\emph{text}
\end{coderesult}

An example:
\begin{resultbox}
\createexample[title={An example that's created on-the-fly provided
that the \cmd{write18} shell escape is enabled},
 description={A document with the word `text' emphasized},
 label={ex:sample}]
{}
{\emph{text}}
\end{resultbox}

\backmatter
\printterms[title={Terms}]
\printcommandoptions{foobarcs}
\printcommonoptions{idx.commonoption}
\printsummary
\printuserguideindex

\end{document}
