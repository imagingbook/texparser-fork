% arara: pdflatex
% arara: makeglossaries
% arara: pdflatex
\documentclass{article}

\usepackage[T1]{fontenc}
\usepackage[colorlinks]{hyperref}
\usepackage[toc,style=index]{glossaries}

\makeglossaries

\newglossaryentry{sample}{name={sample},description={an example}}
\newglossaryentry{child}{parent={sample},
 name={child sample},description={an example sub entry}}

\begin{document}
\tableofcontents

\section{Sample}
\label{sec:sample}
Name value: \glsentryname{sample}.
Text value: \glsentrytext{sample}.
Description value: \glsentrydesc{sample}.

\hypertarget{sampleanchor}{Sample anchor}.

Sentence case: \Glsentryname{sample}.

Is used? \ifglsused{sample}{true}{false}.
A \gls{sample}.
Is used? \ifglsused{sample}{true}{false}.
Next use: \gls{sample}.

\Gls{child}.

\section{Another}
\label{sec:another}

Reference section~\ref{sec:sample} (\ref*{sec:sample}).

\hyperlink{sampleanchor}{Sample hyperlink}.

%\printglossaries
%
%\newenvironment{test}{BEGIN TEST\begin{itemize}}{\end{itemize}END TEST}
%\begin{test}\item Some content\end{test}

\begin{theglossary}
\item Item.
\end{theglossary}
Test.

Other
\end{document}
